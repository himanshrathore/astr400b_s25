\documentclass{article}
\usepackage{graphicx} % Required for inserting images

\title{ASTR400B HW3 Table}
\author{Himansh Rathore}
\date{January 2025}

\begin{document}

\maketitle

\section{Table}

\begin{tabular}{|c|c|c|c|c|c|}
    \hline
    Galaxy Name & Halo Mass & Disk Mass & Bulge Mass & Total & $f_{bar}$\\
    & [$10^{12}$ M$_\odot$] & [$10^{12}$ M$_\odot$] & [$10^{12}$ M$_\odot$] & [$10^{12}$ M$_\odot$] & \\
    \hline
    MW & 1.975 & 0.075 & 0.010 & 2.060 & 0.041 \\
    M31 & 1.921 & 0.120 & 0.019 & 2.060 & 0.067 \\
    M33 & 0.187 & 0.009 & N/A & 0.196 & 0.046 \\
    \hline
    Local Group & & & & 4.316 & 0.054 \\
    \hline
\end{tabular}

\section{Questions}

\begin{enumerate}
    \item Total mass of MW and M31 is the same. DM halo dominates this total mass.
    \item Stellar mass of M31 is larger (1.6 times). M31 more luminous.
    \item DM mass of M31 is 1.028 times the DM mass of MW. Naively, one would expect a linear correlation between the DM mass and stellar mass, so would expect M31 DM mass to be 1.6 times MW. But in reality, the relationship is non-linear and can have a scatter.
    \item MW is 0.041, M31 is 0.067 and M33 is 0.046. The baryon fraction of the universe is more than twice the baryon fraction of the galaxies. There might be baryons elsewhere in the universe, like the IGM.
\end{enumerate}

\end{document}
